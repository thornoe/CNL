\begin{abstract}
In many cases the Multinomial Logit model produces very biased results as the strict assumption of Independence of Irrelevant Alternatives is violated. This paper shows how to solve this by investigating how the Generalized Extreme Value models and in particular the Cross-nested Logit model can be useful for modeling discrete choices in economics and other fields. We present and extend the theoretical background required to understand substitution patterns and other inner workings of the models and derive marginal effects, identification restrictions, and other useful measures.

GEV models like regular Logit models have closed form likelihood functions, and as such lend themselves to regular maximum likelihood estimation. However, as model complexity grows they require increasingly optimized code to be feasibly evaluated and optimized. Likewise the number of constraints and identification issues rise rapidly with complexity, implying that specialized tools are the only time efficient way to estimate parameters in GEV models. %Additionally the complexity of the models severely complicates interpretability.

We simulate data from a GEV process and discuss issues of computation and identification on the basis of these data. We then apply a simple CNL model on data from the Danish DREAM database to test its usefulness in applied econometrics and compare various specifications of GEV models.
\\ \\
\textbf{Keystrokes:} 71.886 \textbf{Standard pages:} 30. \textbf{Contributions:} Thor Donsby Noe: 3.1, 3.2, 4.3, 4.4, 5.1, 5.2, 6.3, 7.1
Kristian Urup Larsen: 3.3, 3.4, 4.1, 4.2,5.3, 6.1, 6.2, 8.1, 8.2
\end{abstract}
